\section{Setup of S88n-interface}
t.b.d.

\subsection{Connection Scheme}
	
%https://www.floodland.nl/aim/info_s88_kabels_en_1.htm


\begin{table}[]
	\caption{S88N pinout and description.}
	\label{tab:my-table}
	\begin{tabular}{|l|l|l|}
		\hline
		\rowcolor[HTML]{9B9B9B} 
		\textbf{RJ45 pin} & \textbf{Colour in UTP cable} & \textbf{S88N Description}             \\ \hline
		1                 & Orange-white                 & +5V (+12V not in this board)          \\ \hline
		2                 & Orange                       & Data                                  \\ \hline
		3                 & Green-white                  & GND                                   \\ \hline
		4                 & Blue                         & Clock                                 \\ \hline
		5                 & Blue-white                   & GND                                   \\ \hline
		6                 & Green                        & Load                                  \\ \hline
		7                 & Brown-white                  & Reset                                 \\ \hline
		8                 & Brown                        & Rail signal (not used in this design) \\ \hline
	\end{tabular}
\end{table}

\subsection{S88UDP installation}
Using updated version from Gbert (original version from Siggi).


Install (to prevent pcap.h compilation error):

sudo apt-get install zlib1g-dev libpcap-dev

Download and install S88ÚDP-rpi:

cd
git clone https://github.com/GBert/railroad
cd railroad/can2udp/src
make

To start the interface: 

sudo ./s88udp-rpi -v -f -c "17,22,23,24" -m 1

Arguments behind option -c are the gpio ports. The amount of S88 modules is set using option -m.

To test if the udp ports are assigned for use by Rocrail:
sudo netstat -autpn | egrep "Proto|157"

The PID "Rocrail" should be displayed.

%% setup bmc
http://www.airspayce.com/mikem/bcm2835/bcm2835-1.63.tar.gz 
tar zxvf bcm2835-1.63.tar.gz
cd bcm2835-1.63
%make && sudo make install
