\section{Setup of the Raspberry Pi}
Prepare an SD-card with Raspberry Pi OS. I have used the version from October 10th 2023 (32-bit) with desktop but without recommended software. My system uses a Raspberry Pi 4B. The advantage of desktop support is that a standalone system can be created by attaching an external monitor, mouse and keyboard. This creates the possibillity of implementing a very small and modular control system.\\

Refer to the \href{https://projects.raspberrypi.org/en/projects/raspberry-pi-setting-up/2}{Raspberry Pi website} for instructions. Follow the instructions to setup the raspberry pi OS. For remote control of the raspberry pi it is possible to enable SSH and/or VNC. Refer to the \href{https://projects.raspberrypi.org/en/projects/raspberry-pi-setting-up/4}{setup instructions} for more information. There is a lot of information available about the setup and realizing SSH or VNC communication on Raspberry Pi support forums.

\subsection{Install BCM2835 v1.73}
Install BCM2835-1.73 so that the pin IO on the Raspberry Pi can be used:
%% setup bmc
\begin{verbatim}
	wget http://www.airspayce.com/mikem/bcm2835/bcm2835-1.73.tar.gz 
	
	tar zxvf bcm2835-1.73.tar.gz
	
	cd bcm2835-1.73
	
	./configure
	
	make
	
	sudo make check
	
	sudo make install
\end{verbatim}

\subsection{Shutdown Button}

% https://howchoo.com/g/mwnlytk3zmm/how-to-add-a-power-button-to-your-raspberry-pi

A button can be connected to the system to simplify the process of shutting the system down. First of all, a pushbutton must be connected between GPIO3 (header pin 5) and GND (e.g. header pin 6). \\

Next, the following script must be installed:

\begin{verbatim}
	git clone https://github.com/Howchoo/pi-power-button.git
	
	./pi-power-button/script/install
\end{verbatim}

Uninstalling the script can be done via:

\begin{verbatim}
	./pi-power-button/script/uninstall
\end{verbatim}

Note: warning about pull-up resistor can be neglected.
